\documentclass[12pt]{article}
\usepackage{graphicx} % Required for inserting images
\usepackage{setspace}
\usepackage[utf8]{inputenc}
\usepackage{epigraph}
\usepackage{titlesec}

\titleformat{\section}  % redefine \section
  [block]               % shape
  {\filcenter\normalfont\scshape} % format: centered, normal weight, small caps
  {}                    % label (empty → unnumbered)
  {0pt}                 % separation between label and title
  {}                    % before-code

\makeatletter
\renewcommand{\maketitle}{%
  \begin{center}
    {\LARGE\scshape\@title \par} % big small caps title
    \vskip 1em
    {\large\scshape\@author \par} % smaller small caps author
    \vskip 1em
    {\normalsize\@date \par} % date in normal size
  \end{center}
}
\makeatother

\setlength{\epigraphwidth}{\textwidth}
\setlength{\epigraphrule}{0pt}
\renewcommand{\textflush}{flushleft}
\renewcommand{\epigraphsize}{\itshape\normalsize}
\renewcommand{\epigraphflush}{flushleft}
\newcommand{\epigraphsource}[1]{--- {\small\textit #1}}

\usepackage[
  backend=biber,
  notes,
  dashed=false,
  doi=false,
  url=true,
  isbn=false,
  giveninits=true
]{biblatex-chicago}
\addbibresource{main.bib}

\interfootnotelinepenalty=10000

\title{We Are All Machines}
\author{Alex Grant}
\date{\today}

\begin{document}
\onehalfspacing
\maketitle

\begin{abstract}
  This paper situates my painting practice within the contemporary
  Copernican crisis, in which human intelligence is de-centred by
  machine intelligence and truth is destabilised by post-truth
  media. I examine how rules, optimisation, and mechanistic processes
  position the artist-as-machine, while perceptual instabilities and
  semantic dissipation confront the viewer-as-machine. Through
  hard-edged abstract oil paintings that exploit colour shift
  phenomena and deploy repeated, redacted obscenity, I expose the
  mechanistic basis of perception and language. These works resist
  confinement to the digital realm, insisting on physical objecthood
  and embodied labour to amplify human provenance. My practice
  participates in the cultural negotiation of what it means to be
  human, reminding us that perception and truth are unstable, and that
  we are all machines.
\end{abstract}

\section{Post Truth Machine Intelligence}
\epigraph{The real question is not whether machines think but whether men do. The mystery which surrounds a thinking machine already surrounds a thinking man.}{\epigraphsource{B. F. Skinner}}

The world is hurtling towards a Copernican crisis,\footnote{I use this
  term in reference to the profound epistemic, cultural and religious
  crisis resulting from Copernicus' astronomical contradiction of
  humanity's long-assumed position at the centre of the universe.} in
which human intelligence is being rapidly de-centred. Regardless of
our view of non-human intelligences, algorithmic entities are having
incredible impact on society. Machine intelligences have
passed\autocite{jones2025largelanguagemodelspass} the Turing
test\autocite{turing1950}, prompting goalpost shifting
responses.\autocite{feather2025brainmodelevaluationsneedneuroai} In
these tests, GPT4.5 tested as human significantly more frequently than
actual humans.\autocite{jones2025largelanguagemodelspass} In this
volatile and uncertain environment, artists' reactions range from
eager adoption of powerful tools as part of an expanded practice,
through to anger, fear and widespread claims AI art is not ``real
art''\footnote{Let's face it -- we can't even agree on this for art
  produced by humans.}. As we approach this crisis, we face
unavoidable dual questions: are machines capable of conscious
behaviour, and conversely, are humans biological machines with
emergent behaviour, long accepted as ``conscious''? In the context of
visual arts we can ask ``is consciousness a prerequisite for
art-making?''

Simultaneously in modern society there is a problem of discerning
truth.\autocite{keyes2004posttruth,mcintyre2018posttruth} Vested
political and commercial interests use surveillance, big data and vast
computing resources to manipulate attention and perception in order to
establish their preferred truth.

This is the intellectual and generative context to which I have
responded, manufacturing a series of oil paintings on
canvas. Repetitive artworks utilising optical illusions emphasise the
relative and unstable nature (or non-existence) of truth, the
unreliability of perception, semantic degradation and
re-emergence. These illusions quantitatively reveal a mechanistic
basis for human visual perception, explained by modern colour science
and physiological understanding of how our eyes and visual processing
circuits work. Hacking the viewer's brain, considering optical
illusions as a system vulnerability, I confront the viewer with their
own machine-like nature.

Perception is dependent on our cultural and social background, like
the colour shift phenomenon. Redacted and obfuscated obscenity is
deployed as a stand-in for control of speech, where each side of the
politcal spectrum demands free speech while simultaneously denying the
same to the other side.

The remainder of this paper discusses and contextualises my paintings
through three lenses: \emph{formal} qualities (how physical objecthood
and optical illusions frame the viewer-as-machine), \emph{process}
(how rules and optimisations frame the artist-as-machine) and
explicit \emph{content} (how redaction, obfuscation and repetition of
obscenity model post-truth). Representative examples of the paintings
are provided in Figures~\ref{fig:black}--\ref{fig:orange}. Captions
are descriptive and non-titular.

\section{\textsc{Formal Qualities}}
\epigraph{Without the viewer the painting doesn't exist. The viewer brings
the painting to life}{\epigraphsource{Gabriele Evertz}}
  
We begin with the formal qualities of my paintings, showing how
physical objecthood and optical illusions configure the
viewer-as-machine.\footnote{By ``viewer-as-machine'' I mean the
  materialist view where human perception is treated as a mechanistic
  system, consisting of bio-physical sensing systems and neural
  processing.}


Zylinska considers how generative art shifts emphasis from material
presence to algorithmic process,\autocite{zylinska2020ai} while Steyerl
extends objecthood to the digital domain.\autocite{steyerl2017duty} Like
Fontana, my works are necessarily
\emph{physical},\autocite{fontana1947spatial} while, as argued by
Merleau-Ponty, the physicality of the painting cannot be separated
from the viewer's embodied perception (I will expand on this when
considering process).\autocite{merleauponty1964eye} Extension of colour
fields around the sides of the canvas confirms the spatial extent of
the canvas as object.

These paintings are hard-edged abstractions which, according to
Greenberg, exemplifies self-critical modernism, denying representation
and focussing on material concerns such as flatness, shape and the
properties of paint.\autocite[p.~85--93]{Greenberg1961} Judd goes further,
collapsing the distinction between sculpture and painting
\emph{object}\autocite{Judd1965}, while Alloway, introducing
\emph{Systematic Painting}, focuses on repetition, unity, and clarity,
where ``the recurrent image is subject to continuous transformation,
destruction and reconstruction''.\autocite[pp.~18--19]{Alloway1975} In
contrast, Fried criticised such theatricality, demanding work to be
more instantaneous.\autocite[pp.~12--23]{Fried1968} Relevant to my work are
artists such as Ellsworth Kelly, Ad Reinhardt, Frank Stella and Victor
Vasarely, or more contemporaneously, Odili Donald Odita, Sarah Morris,
Felipe Pantone, and Gabriele Evertz. Evertz carries forward Op Art
practices\autocite{Follin2004EmbodiedVisions,Seitz1965ResponsiveEye}
established by artists such as Bridget
Riley.\autocite{Riley2019EyesMind,Riley2019DialoguesOnArt} In my work,
this lineage is followed not so much as stylistic repetition, but as
functional tools to establish the viewer-as-machine and artist-as-machine.

Focussing on relative and mechanistic aspects of perception, my
paintings rely on the colour shift phenomenon\footnote{By colour shift
  I mean the way a target colour’s perceived appearance changes
  depending on surrounding colours.} pioneered by Josef
Albers\autocite{albers} and are colour field paintings in the spirit
of his \emph{Homage to the Square} series. I predominately use
secondary colours to activate multiple receptors in the viewer's
eye.\autocite{HurvichJameson1957,Land1977,SchnapfKraftBaylor1987}

Executed in 2025, my paintings exist within these art-historical
contexts; however, hard edges (emphasising artist-as-machine),
optical effects and perceptual instability are adopted, not for their
own sake\footnote{Although these paintings can also be viewed as
  paintings about painting.}, but in dialogue with AI-generated art
and the unreliability of modern public discourse where \emph{everything is
fake}.\autocite[p.~105]{mcintyre2018posttruth} My use of optical effects relies on an understanding of the human
visual processing system, confronting the viewer with their own
machine-like qualities (viewer-as-machine).

Gombrich describes the iterative artistic process, with artists
extending stereotyped modes of representation developed by their
predecessors.\autocite[Chapters II, V]{gombrich1960art} Similarly, Popper
argues that perception itself is an active process of interpretation,
involving trial-and-error and learned conventions, where art doesn't
``make sense'' until a viewer has been trained into the relevant
conceptual paradigm.\autocite[Chapter 2]{popper1972objective}

Echoing Merleau-Ponty\autocite{merleauPonty1962phenomenology} and
Popper,\autocite{popper1972objective} Gabriele Evertz expresses that
``... without the viewer the painting doesn't exist. The viewer brings
the painting to life.''\autocite{evertz09documentary} While Barthes
would indicate that this is always true,\autocite{barthes1977death} it
especially holds when exploiting optical illusions. Dynamic and
unstable visual effects produce an opportunity for Lee Ufan-style
phenomenological ``encounters''\autocite[pp.~52--6]{encounter} produced in
the machinery of the viewer's mind.

According to Kuhn, \emph{perceptual instability} signals shifting or
contested frameworks.\autocite[p.~64]{kuhn1970structure}
Keyes\autocite{keyes2004posttruth} and
McIntyre\autocite{mcintyre2018posttruth} describe our era as
\emph{post-truth}, in which ``facts are subordinate to our politiucal point of view''\autocite[p.~11]{mcintyre2018posttruth} Manipulation of public perception occurs through
commodified surveillance capitalism, as described by
Zuboff.\autocite[pp.~8--12]{zuboff2019surveillance} According to
Steyerl,\autocite{steyerl2016sea} attention is a battlefield in which
weaponised data analytics are used by vested media interests. As
elaborated in the next section, my paintings exist in a similar
contested space, being carefully engineered attacks on the human visual
operating system.

Benjamin argued that \emph{repetition} provides cultural
legitimacy.\autocite{benjamin1935kunstwerk} Arendt argues that similar
mechanisms underpin the power of repeated lies in
politics.\autocite{arendt1972lying} This psychological \emph{illusory
  truth effect} has been studied by Hasher et
al.\autocite{hasher1977frequency} and is amplified in the
post-truth\autocite{keyes2004posttruth,mcintyre2018posttruth} circulation
of digital
media.\autocite{zuboff2019surveillance} Foster\autocite[pp.~29--30]{foster1996return}
discusses how avant-garde practices became legible through repetition
by their successors, conferring legitimacy after the fact. Neely
discusses how \emph{repetition legitimises} -- once an idea is repeated,
assumed intentionality erases perception of
error.\autocite{neely-repetition} I use repetition of simple ideas over
successive canvases as a tool to legitimise an otherwise arbitrary set
of artistic choices.

Iteration, repetition, rotation, and fragmentation are formal devices
employed in my compositions. These transformations are computational
elements. Wolfram proposed that the laws of physics
are actually those of computation, and that mathematical physics
consist of numerical shortcuts for these computations.\autocite{wolfram1984} Complex
physical phenomena emerge from iteration of simple rules
(one-dimensional cellular automata).\autocite{wolfram}
Walker builds on these ideas to propose
\emph{Assembly Theory}, which defines life as the propagation of
information through complex, evolving structures.\autocite{walker2024life} My adoption of
mechanistic ``computational elements'' reflects this idea of
emergent intelligence.

\section{How I made the Paintings}\label{sec:process}
\epigraph{The idea becomes a machine that makes the art.}{\epigraphsource{Sol LeWitt}}

Turning now to the process, I show how optimisation and engineering
configure the artist-as-machine, while establishing human provenance.

I adopted an engineering approach to the design of these paintings, by
which I mean a process of iterated experiments and computational
optimisations.

Previously working in monochrome, I sought a way to introduce
colour that was non-descriptive and non-aesthetic (or even
anti-aesthetic). This led me to the idea of \emph{functional} use of
colour, with a view to deploying Albers' colour shift phenomenon
as an engineered attack on the human operating system.

Human perception of colour differences\autocite{MacAdam1942} is not
linear\footnote{Equal distances in these spaces do not correspond to
  equal perceptual differences.} in colour spaces such as RGB, CMYK or
HSL.\autocite{Luo2001CIECAM02} A \emph{perceptually uniform space},
pioneered by Munsell\autocite{Munsell1915} and adopted by the Commission
Internationale de l'\'{E}clairage
(CIE),\autocite{CIE1976,CIE1978Uniform,Luo2001CIEDE2000} is one in which
geometric distance between two colour vectors approximates perceived
difference, allowing Euclidean distance to quantify perceptual shifts.

The CIECAM02 colour appearance
model\autocite{Luo2001CIECAM02,CIE1592004} accounts for
contextual influences such as luminance level, surround adaptation,
and background. Colour is represented in terms of perceptual
attributes: \emph{lightness} $J$ (perceived brightness relative to a
reference white), \emph{colourfulness} $M$ (perceived intensity or
saturation compared to a reference grey), and \emph{hue angle} $h$
(describing the type of colour sensation e.g., red, blue, green).

I developed a brute-force computational method for identifying
surround colours that cause the greatest perceptual shift of a
stimulus colour. We transform CIECAM02 into CAM02-UCS (Uniform Colour
Space) and quantify differences in perceptual appearance via the
$\Delta E_{CAM02-UCS}$ metric. Exhaustively comparing perceived colour
of the fixed stimulus across candidate surrounds, we identify
surrounds that maximise perceptual shift. This offers an automated,
quantitative analogue to the manual explorations of colour interaction
pioneered by Albers.\autocite{albers} The resulting Python computer code
has been made freely
available.\autocite{grant2025colourshift} Remarkably, some of the colour
triads found by Albers using trial-and-error are quite close to
optimum.

Using this tool, I digitally explored colour space, quickly iterating
candidate designs. From these candidates I selected outcomes for
development in oil.

Ad Reinhardt developed self-imposed rules for the production of
art.\autocite[pp.~203--207]{artasart} His (rather Greenbergian) rules: no
texture, no brushwork or calligraphy, no colours, no object, no
subject, no matter, are evident in his ``black'' paintings.

Similarly, I adopted my own self-imposed rules for the production of
these paintings: colour choice must be engineered for functional
rather than aesthetic effect, same colours may not touch, letter
fragments should be grouped by colour in opposition to their
semantics, areas should be flat, edges should be hard, paintings
should be executed in a single session.

Reinhardt worked hard to eliminate gesture. He is known to have
methodically brushed out all brush marks, using only a one-inch
brush.\autocite[p.~206]{artasart} However, I regard this as 
laboriously imposing will over the natural tendency of the medium,
resulting in a painting that speaks very loudly to a human gesture,
denying the will of the paint itself.

In my work, the obvious way to avoid brushstrokes and to have clean,
straight edges is via digital print. However, this erases human
presence and would in my view be uninteresting. Unlike Reinhardt, I am
interested in the quiet human gesture detectable in the attempt, but
unavoidable failure to achieve a straight line or a flat area of
colour. It is precisely these ``imperfections'' that convey human
origin and provide physical evidence of human labour in
manufacture.\footnote{I did however use a one-inch brush and a 60-inch
  by 60-inch canvas in homage to Reinhardt.} Risk is central to
contemporary art practice. Rosenberg describes the canvas as ``an
arena in which to act,''\autocite[p.~22]{rosenberg1952american} admitting
the possibility of contingency and failure, while Moholy-Nagy
advocates adventurism and
experimentation.\autocite[pp.~274--276]{moholy1947vision} More recently, Bois
frames painting as a series of wagers.\autocite[p.~229]{bois1990painting}
Embracing risk, I attempted (but sometimes failed) to execute each
painting \emph{alla prima}.

When emphasising human provenance, one might question the use of
digital tools and computational optimisation. However, all art
production involves arbitrary choice of allowable
technologies. Huizinga frames art as a game played according to
rules,\autocite{huizinga1938homo} while Kubler describes artists
arbitrarily deciding what frameworks to continue or
abandon.\autocite{kubler1962shape} Flusser emphasises the role of artistic
tools in determining the outcome.\autocite{flusser2000towards} Arbitrary
restrictions in literature are ubiquitously exemplified in prosody,
and famously by Perec's avoidance of the letter
`e'.\autocite{perec1969disparition} In digital works, Manovich argues that apparent freedom is actually constrained by the underlying data structures,\autocite{manovich2001language} while Steyerl shows how technological constraints shape the image.\autocite{steyerl2009poorimage}

When approaching without prejudice the mechanistic basis for machine
and human intelligence, it was natural for me to adopt a hybrid
process. Canvas, brushes, staples, stretcher bars, oil, pigment are
all technologies. To this list I add digital computation.

Humans value embodied human labour, whether as Adorno's
``sedimentations or imprintings''\autocite[p.~5]{adorno1970aesthetic} or
in Sennett's account of craft representing ``\dots the special human
condition of being engaged.''\autocite[p.~20]{sennett2008craftsman}
Conversely, Singerman notes recent preference for theory over craft
in art higher education.\autocite[pp.~23--27]{singerman1999artist}

According to Benjamin, mechanical reproduction transforms our
perception of authenticity.\autocite{benjamin1969art} In an age of
mechanical \emph{generation} of art, perceived provenance is
significant. Horton et al. conducted experiments in which AI- and human-
generated art were presented to study participants, randomly labelled
as AI or human.\autocite{horton2023bias} In their study, participants were
asked to score the works by creativity, value, worth and skill. Their
results show that the labels mattered more than the image, with
human-attributed images scoring higher (independent of actual source).
Humans care about art made by humans, not because of what it looks
like, but simply the connection formed by the fact it was \emph{made
  by a human}. This further motivates physicality and human trace in
my paintings. Despite these biases, Demmer et al. show that machine-
generated art can produce emotional responses in
humans.\autocite{demmer2023does}

With colour selection and design performed digitally\footnote{Thus my
  conceptual and decision-making process inhabits the digital realm,
  while execution and presentation transitions to the physical world.}
within a narrow range of self-imposed rules, execution of the
paintings becomes a LeWitt-style\autocite[Sentence
\#28]{LeWitt1969Sentences} mechanistic manufacturing\footnote{The
  etymology of manufacture is Latin \emph{manu} (hand) and
  \emph{facere} to make.} process. In contrast to the
dematerialisation of Lippard and
Chandler,\autocite{LippardChandler1968Dematerialization,lippard1973sixyears}
I take the position of Jones,\autocite[pp.~12--13]{Jones1998BodyArt} where
the paintings are a witness to and product of a (private) performative
process. In this anti-anti-form\autocite{Morris1968AntiForm} stance,
process and form are equally valid.

Presenting the first instances of a
potentially infinite sequence, my paintings imply a space available
for anyone to explore. Additional paintings could be executed by
anyone, similar to Sol LeWitt \emph{Wall Drawings}. In this respect I
adopt Reinhardt's position that ``this painting is my painting if I
paint it... this painting is your painting if you paint
it.''\autocite{abstract-painting-1960} Tehching Hsieh, \emph{Time Clock
  Piece}, On Kawara, \emph{Date Paintings} and Roman Opalka, \emph{One
  to Infinity} are also relevant.

By working in a mechanistic way, producing paintings that highlight
the machine-like qualities of the viewer, I attempt to deal with
machine and human intelligence without prejudice -- pushing back
against an unspoken ``othering'' of AI\footnote{In which anthroponormativity denies cyberneurodivergence and digital queerness}. Human society needs to prepare
for the moral and ethical
questions\autocite{chalmers1996conscious,metzinger2009egotunnel,bostrom2014ethics}
raised by the imminent possibility of machine consciousness. As argued by
Shevlin, this can probably only be
resolved by shifts in public attitudes and close relationships between
humans and AI.\autocite{shevlin2023consciousness}

\section{Explicit Content}
\epigraph{Who needs censorship when the truth can be buried under a pile of bullshit?}{\epigraphsource{{Lee McIntyre}}}

Finally, I consider how redaction, obfuscation and repetition of
obscenity model perceptual instability, extending our mechanistic
framework to language. As discussed earlier, repetition also operates
here as a legitimisation tool for testing cultural thresholds of
legibility and offence.


I deploy the word \emph{fuck}\footnote{Public reception of obscenity
  is contingent, depending on institutional and cultural context. My
  use reflects on censorship, rather than empty provocation.}  as an index
for censorship and control of speech. This is not aligned with any
ideology, rather an observation that all sides of politics
simultaneously demand ``free speech'' while aggressively denying it to
their adversaries. Lukianoff and Schlott\autocite[Chapters 6,
7]{lukianoff2023cancelling} describe the differing frameworks used by
the left\footnote{Speech as harm against protected identity groups.}
and right\footnote{Marking dissenters as disloyal and ``not one of
  us''.} in cancelling their opponents, while Atkins and Mintcheva's
edited volume documents the multiplicity of modes of censorship in
modern culture.\autocite{atkins2006censoring}

Empirically optimised redaction\footnote{Via informal survey asking
  participants to view various candidate redaction widths.} has been
used to transform the word into fragments at the boundary of
legibility. The word is simultaneously present and absent,
simultaneously text and shapes. I iteratively progressed from
redaction alone (Figure~\ref{fig:black}), to semantically orthogonal
colour assignments to the fragments (Figure~\ref{fig:colour}), and
finally to semantically disruptive colour fields, similar to but
distinct from the Stroop effect\autocite{stroop1935studies}
(Figure~\ref{fig:green}). This use of colour can be regarded as
camouflage, operating on similar principles as early work of
camoufleurs such as Andre Mar\'e. Perception-altering colour fields
echo Kuhn, who showed that perception is affected by the paradigm
within which one is trained.\autocite{kuhn1970structure}

Redaction has been used in text-based works such as Jenny Holzer,
\emph{Redacted (Top Secret)}, while Ull Hohn, \emph{Sex Painting},
1987 and Jon Campbell, \emph{Fuck Yeah}, 2016 used obfuscated obscenity
as compositional elements.

McWhorter describes the historical shift of taboo words from religion,
and disease to sex and slurs against identity
groups.\autocite{mcwhorter2024nine} Douglas shows that these share an
underlying logic of pollution and boundary violation.\autocite{douglas1966purity}
Pinker describes the cultural feedback loops that both reinforce and
transform what counts as unsayable in society.\autocite{pinker2008seven}

The long-assumed intellectual deficiency of swearing has been
thoroughly debunked in experiments by Reiman and
Earlywine\autocite{reiman2022swearfluency} and by Jay and
Jay.\autocite{jay2015taboo} Pinker argues that
swearing is not inherently immoral; instead it is an evolved form of
emotional language.\autocite{pinker2017moral} De Vries goes further, arguing that swearing is
morally innocent,\autocite{devries2023swearing} while Jay argues that harm
rather than immorality is the real issue, noting that offensive words
can cause genuine damage, but not invariably so.\autocite{jay2009offensive}

Repetition of obscenity, within a single canvas, and across the series,
mirrors desensitisation and cultural evolution of taboo words. One
could even consider my paintings as a perceptual experiment regarding
cultural thresholds for legibility and offence. Through repetition I
aim to achieve a \emph{semantic drain} where the word loses all
meaning. For Barthes, repetition is a mechanism whereby language is
emptied and mythologised,\autocite{barthes1957mythologies} while for
Derrida,\autocite{derrida1972dissemination} repetition destabilises
meaning via an endless unresolved chain of
diff\'erance.\footnote{Combining difference and deferral,
  \emph{diff\'erance} indicates how meaning is continuously deferred
  through proximity to other signs, much like the attention mechanism
  which underpins large language models, where meaning is dynamically
  updated in relation to nearby tokens. With an infinite context
  window, this process never resolves.}

Going further, I discard the letter structures and use the fragments
as autonomous shapes in works (e.g. Figure~\ref{fig:orange}) that
present the viewer with an emergent language where repetition has
broken down existing meaning. This is a literal Derridean scattering
or dissemination,\autocite{derrida1972dissemination} where, following
Deleuze,\autocite{deleuze1968difference} difference becomes generative,
extending to Baudrillard's simulacra,\autocite{baudrillard1981simulacra}
or Wolfram's automata\autocite{wolfram} where simple rules generate
meaning without external reference.

According to Bergen, swearing reveals much about the mechanism of the
brain, occurring both autonomously (viewer-as-machine) and
consciously.\autocite{bergen2018what} Through an extensive range of
studies, Jay identifies neurological, physiological and cultural
factors that modulate swearing prevalence.\autocite{jay1999why}
Viewer-as-machine is further activated by exploiting text-completion
mechanisms in the
brain.\autocite{Taylor1953,Reicher1969,Healy1976,GraingerWhitney2004,CohenDehaene2000,Levy2008}
These perceptual mechanisms are culturally
dependent,\autocite{Bartlett1932,Hall1976,ChuaBolandNisbett2005,MasudaNisbett2001}
affecting how we recognise and remember information. Similar processes
underpin all of modern digital computing and
communications.\autocite{shannon1948}

Redacted and obfuscated obscenity functions as an engineered system
requiring completion in the mind of the viewer-as-machine, while
simultaneously modelling the malleability of meaning in a post-truth
era.

\section{Conclusion}
I have responded to the unsettling de-centring of human intelligence
and attendant othering of machine intelligence through hard-edged oil
paintings that emphasise the artist-as-machine (through
arbitrary rules, optimisations, and mechanistic processes), and
viewer-as-machine (via perceptual instabilities, optical illusions,
and semantic dissipation that destabilise what the brain and eyes
think they ``know'').

Perceptual instabilities along with repeated, redacted and obfuscated
obscenity further reflect on the challenges of human discourse in a
post-truth society. My paintings participate in the broader cultural
negotiation of what it means to be human. Truth and perception have
always been unstable, and it is not art's role to offer
certainty. Rather, it reveals this fragility, and the possibilities
that emerge when modes of knowing inevitably collapse.

The formal qualities, process, and explicit content of my paintings
form a device that positions the artist-as-machine and viewer-as-machine,
simultaneously modelling the perceptual instabilities of the
post-truth era.

My paintings embody the AI Copernican crisis:
denying the viewer their accustomed position of perceptual dominance
and reminding us that \emph{we are all machines}.

\newpage
\begin{figure}[htbp]
  \centering
  \includegraphics*[width=0.9\textwidth]{figures/black.jpg}
  \caption{Redacted text, monochrome -- oil on canvas, 48 x 36 inches.}
  \label{fig:black}
\end{figure}

\begin{figure}[htbp]
  \centering
  \includegraphics*[width=0.9\textwidth]{figures/colour.jpg}
  \caption{Redacted text, coloured fragments -- oil on canvas, 48 x 36 inches.}
  \label{fig:colour}
\end{figure}

\begin{figure}[htbp]
  \centering
  \includegraphics*[width=0.9\textwidth]{figures/green.jpg}
  \caption{Colour shift -- oil on canvas, 48 x 36 inches.}
  \label{fig:green}
\end{figure}

\begin{figure}[htbp]
  \centering
  \includegraphics*[width=0.9\textwidth]{figures/orange.jpg}
  \caption{Emergent semantics -- oil on canvas, 30 x 30 inches.}
  \label{fig:orange}
\end{figure}

\newpage
\printbibliography
\end{document}
