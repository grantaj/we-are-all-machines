\documentclass[12pt]{article}
\usepackage{graphicx} % Required for inserting images
\usepackage{setspace}
\usepackage[utf8]{inputenc}
\usepackage[%
  titleformat=italic,% Titles in italic 
  titleformat=commasep,% A comma between athors and title 
  titleformat=all,% Always show a title (or a short title)
  commabeforerest,% A comma after title 
  ibidem=strict,% 
  citefull=first,% The first citing in full form 
  oxford,% The oxford style
  super,% Footnotes 
  see
]{jurabib}

\title{We Are All Machines}
\author{Alex Grant}
\date{\today}

\begin{document}
\onehalfspacing
\maketitle

The world is hurtling towards a Copernican crisis, in which human
intelligence is being rapidly de-centred. Regardless of our view of
non-human intelligences, algorithmic entities are having incredible
impact on society. Machine intelligences have
passed~\cite{jones2025largelanguagemodelspass} the Turing
test~\cite{turing1950}, prompting goalpost shifting
responses~\cite{feather2025brainmodelevaluationsneedneuroai}. In these
tests, GPT4.5 tested as human significantly more frequently than
actual humans. In this volatile and uncertain environment, artists'
reactions range from eager adoption of powerful tools as part of an
expanded practice, through to anger, fear and widespread claims that
AI art is not ``real art''\footnote{Let's face it -- we can't even
  agree on this for art produced by humans.}. As we approach this
crisis, we face unavoidable dual questions: are machines capable
of conscious behaviour, and conversely, are humans biological machines
with emergent behaviour that has been long been accepted as
``conscious''? In the context of visual arts we can ask ``is
consciousness a prerequisite for art-making?''

The emergence of machine intelligence reflexively prompts questions regarding human intelligence, and the extent to which we are all machines. Optical illusions (such as the colour shift phenomenon) quantitatively reveal a mechanistic basis for human visual perception, and are explained by understanding of how our eyes and visual processing circuits work. By using these illusions (I think of this as brain hacking, and optical illusions as a system vulnerability), I confront the viewer with their own machine-like nature.

 Contemporaneously, in modern society there is a problem of discerning truth~\cite{keyes2004posttruth,mcintyre2018posttruth}.  Vested interests on all sides of politics and commerce deliberately and consciously use surveillance, big data and vast computing resources  to manipulate attention, perception, and to project their preferred truth. 
 
 In a world of fake news, our default response is to regard the inherent unreliability of everything we encounter on the internet. This is the intellectual and generative context to which I have responded,  manufacturing a series of oil paintings on canvas. 
 
 By presenting the viewer with repetitive artworks utilising optical illusions, I emphasise the relative and unstable nature (or even nonexistence) of truth, the unreliability of our own perception, semantic degradation and re-emergence. Perception is dependent on our cultural and social background, just like the colour shift phenomenon.  Redacted and obfuscated obscenity is deployed as a stand-in for control of speech, where each side demands free speech while simultaneously denying the same to the other side.

The remainder of this essay discusses and contextualises these paintings within a framework of \emph{formal} qualities, \emph{process} and \emph{content}.

\section{Formal Qualities}
Zylinkska considers how generative art shifts emphasis from material presence to algorithmic process~\cite{zylinska2020ai}. Steyerl extends objecthood to the digital domain~\cite{steyerl2017duty}. In response, my works are necessarily \emph{physical}~\cite{fontana1947spatial,merleauponty1964eye}. Extension of the colour fields around the sides of the canvas emphasises the spatial extent of canvas as object. 

These paintings are hard-edged abstractions~\cite{Greenberg1961,Alloway1975,Judd1965,Fried1968} calling back to artists such as Ellsworth Kelly, Ad Reinhardt, Frank Stella and Victor Vasarely, or more contemporaneously, artists such as Odili Donald Odita, Sarah Morris, Felipe Pantone, and Gabriele Evertz. Of particular relevance, Evertz carries forward op art practices~\cite{Follin2004EmbodiedVisions,Riley2019EyesMind,Riley2019DialoguesOnArt,Seitz1965ResponsiveEye} established by artists such as Bridget Riley. Emphasising the relative and mechanistic aspects of perception, my paintings rely on the colour shift phenomenon pioneered by Josef Albers~\cite{albers} and are colour field paintings in the spirit of Albers' \emph{Homage to the Square} series. I predominately use secondary colours to  activate multiple receptors in the viewer's eye~\cite{HurvichJameson1957,Land1977,SchnapfKraftBaylor1987}.

Executed in 2025, my paintings exist within these art historical contexts, however hard edges (emphasising artist-as-machine) and perceptual optical effects (perceptual instability) are respectively adopted, not just for their own sake\footnote{Although these paintings can also be viewed as paintings about painting.}, but in dialogue with AI generated art and the unreliability of modern public discourse (everything is fake). My use of optical effects relies on understanding of the human visual processing system, confronting the viewer with their own machine like qualities (viewer-as-machine). 

Perception as an active process of interpretation~\cite{popper1972objective}, involving trial-and-error and learned conventions -- proceeding recursively with each artist extending stereotyped modes of representation utilised by previous artists~\cite{gombrich1960art}.  Art doesn't ``make sense'' until a viewer has been trained into the relevant conceptual paradigm.

Echoing Merleau-Ponty~\cite{merleauPonty1962phenomenology} and Popper~\cite{popper1972objective}, Gabriele Evertz expresses that ``... without the viewer the painting doesn't exist. The viewer brings the painting to life.''~\cite{evertz09documentary}. While this always true to some extent~\cite{barthes1977death}, this especially holds in the case of art exploiting optical effects and illusions. The dynamic and unstable effect is an ``encounter''~\cite{encounter} produced in the machine that is the viewer's mind. 

\emph{Perceptual instability} signals shifting or contested frameworks~\cite{kuhn1970structure}. In a post-truth~\cite{keyes2004posttruth,mcintyre2018posttruth} age, large-scale manipulation of public perception via surveillance capitalism, weaponized data analytics, and vested media interests can be viewed as engineered attack on human attention~\cite{zuboff2019surveillance, steyerl2016sea}. My paintings are a carefully engineered attack on the human visual operating system (see Section~\ref{sec:process}). 

Benjamin~\cite{benjamin1935kunstwerk} argued that \emph{repetition} provides cultural legitimacy to the image. Arendt argues that a similar mechanism underpins the  power of repeated lies in politics~\cite{arendt1972lying}. This \emph{illusory truth effect} has been studied in psychology~\cite{hasher1977frequency} and is amplified in the post-truth~\cite{keyes2004posttruth,mcintyre2018posttruth} circulation of digital media~\cite{zuboff2019surveillance}. I use repetition of simple ideas over successive canvases simultaneously in reference to these ideas, and as a tools to legitimise~\cite{foster1996return,neely-repetition} an otherwise arbitrary set of artistic choices. 

Iteration, repetition, rotation, and fragmentation are formal devices employed in the composition of the paintings. These transformations can be viewed as computational elements. Wolfram~\cite{wolfram1984} proposed that the laws of physics are actually those of computation, and that mathematical laws of physics are simply numerical shortcuts for these computations. He further proposed that there are \emph{irreducible} physical processes for which the only way to determine the future state is to run forward the state of the universe itself. This was developed further in \emph{A New Kind of Science}~\cite{wolfram}, which explores in depth the emergence of complex physical phenomena from iteration of very simple rule sets (one dimensional cellular automata). Sara Walker~\cite{walker2024life} builds on these ideas to propose \emph{Assembly Theory} which defines life as the propagation of information through complex, evolving structures. My adoption of mechanistic ``computational elements'' reflects on this idea of emergent intelligence.

\section{How I made the Paintings}\label{sec:process}
I adopted an engineering approach to the design of these paintings. By engineering approach, I mean a process of iterated experiments, combined with computational optimisations. 

Previously working in monochrome, I was seeking a way to introduce colour that was non-descriptive and non-aesthetic (or even anti-aesthetic). This led me to the idea of \emph{functional} use of colour, with a view to deploying the Albers' colour shift phenomenon as an engineered attack on the human operating system.

Human perception of colour differences~\cite{MacAdam1942} is not linear in most common colour spaces such as sRGB or CIE 1931 XYZ~\cite{Luo2001CIECAM02}. That is, equal distances in these spaces do not correspond to equal perceptual differences. A \emph{perceptually uniform space}, pioneered by Munsell~\cite{Munsell1915} and adopted by the Commission Internationale de l'\'{E}clairage (CIE) CIE~\cite{CIE1976,CIE1978Uniform,Luo2001CIEDE2000} 
is one in which the geometric distance between two colour vectors approximates perceived difference. This allows us to use Euclidean distance to meaningfully compare perceptual shifts.

CIECAM02 accounts for contextual influences such as luminance level, surround adaptation, and background colour. Each colour is represented in terms of perceptual attributes: lightness $J$ (perceived brightness of a colour relative to a reference white.), colourfulness $M$ (perceived intensity or saturation of colour compared to a reference grey), hue angle $h$ (describing the type of colour sensation e.g., red, blue, green).

I developed a computational method for identifying surround colours that cause the greatest perceptual shift in the appearance of a fixed stimulus colour. The method simulates perceptual colour appearance using the CIECAM02 colour appearance model~\cite{Luo2001CIECAM02,CIE1592004} and quantifies differences in appearance via the $\Delta E_{CAM02-UCS}$ metric. To compare colours perceptually, we transform CIECAM02 into the CAM02-UCS (Uniform Colour Space). By exhaustively comparing the perceived colour of a fixed stimulus across a large set of candidate surrounds, we identify surrounds that maximize perceptual shift. This system offers an automated, quantitative analogue to the manual explorations of contextual colour interaction pioneered by Josef Albers~\cite{albers}. The resulting computer code, written in python, has been made freely available~\cite{grant2025colourshift}. Remarkably, some of the colour triads found by Albers using a trial-and-error approach are quite close to optimum. 

Using this tool, I digitally explored colour space, quickly iterating candidate designs. From these candidates I then selected outcomes for development in oil. This selection process was human -- consisting of me simply looking and choosing.

In \emph{Twelve Rules for a New Academy}~\cite[p. 203-7]{artasart}, Reinhardt developed self-imposed rules for the production of art . These ``rules'' (no texture, no brushwork or calligraphy, no colors, no object, no subject, no matter) are evident in Reinhardt's so-called ``black'' paintings. 

Similarly, I have adopted my own self-imposed rules for the production of these paintings:
colour choice must be engineered for functional rather than aesthetic effect, same colours may not touch, letter fragments should be grouped by colour in opposition to their semantics, areas should be flat, edges should be hard, paintings should be executed in a single session.

Reinhardt worked hard to eliminate gesture from this painting (he is known to have methodically brushed out all brush marks, using only a 1 inch brush~\cite[p. 206]{artasart}). However, I regard this as self-contradictory, viewing this as the artist laboriously imposing his will over the natural tendency of the medium, resulting in a painting that speaks very loudly to a human gesture - that of denying the will of the paint itself. In my work, the obvious way to avoid brushstrokes and to have straight lines is to simply produce a digital print. However this erases the human presence and would in my view be uninteresting. I am much more interested in the quiet human gesture detectable in the attempt, but failure to achieve a straight line or a flat area of colour. It is precisely these ``imperfections'' that convey human origin and provide physical evidence of human labour in manufacture.\footnote{I did however use a one inch brush and a 60 inch by 60 inch canvas in homage to Reinhardt.} To amplify this effect I attempted (but sometimes failed) to execute each painting \emph{alla prima}, embracing risk of failure.\cite{rosenberg1952american,moholy1947vision,bois1990painting}.

When emphasising human provenance, one might question the use of digital tools and computational optimisation. However all art production involves arbitrary choice of allowable technologies~\cite{huizinga1938homo,kubler1962shape,flusser2000towards,manovich2001language,steyerl2009poorimage,perec1969disparition}.  When approaching without prejudice the mechanistic basis for machine and human intelligence, it was natural for me to adopt a hybrid process.  Canvas, brushes, staples, stretcher bars, oil, pigment are all technologies. To this list I add digital computation.

According to Benjamin, mechanical reproduction transforms our perception of authenticity~\cite{benjamin1969art}. In our new age of mechanical \emph{generation} of art, perceived provenance is significant. Horton et al.~\cite{horton2023bias} conducted experiments in which AI and  human generated art were presented to study participants, randomly labelled as AI or human. Participants were asked to score the works in categories such as creativity, value, worth and skill. Their results show that the labels mattered more than the image, with human-attributed images scoring higher (independent of actual source).  Human care about art made by humans, not because of what it looks like, but simply the connection formed by the fact it was \emph{made by a human}. Humans value embodied human labour~\cite{adorno1970aesthetic,singerman1999artist,sennett2008craftsman}. In another study, Demmer et al.~\cite{demmer2023does} show that machine generated art can produce emotional responses in humans. 

With colour selection and design performed digitally\footnote{Thus my conceptual and decision making process inhabits the digital realm, while execution and presentation transitions to the physical world.} within a narrow range of allowable self-imposed rules, execution of the paintings becomes a mechanistic manufacturing process or performance~\cite[Sentence \#28]{LeWitt1969Sentences}\cite{Jones1998BodyArt,LippardChandler1968Dematerialization,lippard1973sixyears}. Taking an anti-anti-form~\cite{Morris1968AntiForm} stance, the process and form are equaliy valid.  Presenting the first instances of a sequence, my paintings imply a space available for anyone to explore. With the rules in place, additional paintings could be executed by anyone, similar to Sol LeWitt \emph{Wall Drawings}. In this respect I adopt Reinhardt's position that ``this painting is my painting if I paint it... this painting is your painting if you paint it''~\cite{abstract-painting-1960}. Tehching Hsieh, \emph{Time Clock Piece}, On Kawara, \emph{Date Paintings} and Roman Opalka, \emph{One to Infinity} are also relevant.

By working in a mechanistic way, producing paintings that highlight the machine-like qualities of the viewer, I attempt to deal with machine and human intelligence without prejudice -- pushing back against an unspoken ``othering'' of AI. Human society needs to prepare for the significant and difficult moral and ethical questions~\cite{chalmers1996conscious,metzinger2009egotunnel,bostrom2014ethics} raised by the possibility of machine consciousness. As argued by Shelvin~\cite{shevlin2023consciousness}, this can probably only be resolved by shifts in public attitudes and close relationships between humans and AI. 

\section{Explicit Content}
In this series of paintings I deploy the word \emph{fuck} as a stand-in for the increasing censorship and control of speech\cite{atkins2006censoring}. This is not aligned with any ideology, rather an observation on all sides of the political spectrum simultaneously demanding ``free speech'' while denying it to their adversaries\cite{lukianoff2023cancelling}.

Empirically optimised redaction\footnote{ (Via informal survey asking participants to view various widths of redaction.} has been used to transform the work into fragments that hover at the boundary of legibility. I iteratively progressed from redaction alone through to semantically orthogonal colour assignments to the fragments and then to semantically disruptive colour fields (similar but distinct from the Stroop effect~\cite{stroop1935studies}). This use of colour can be regarded as camouflage, operating on similar principles as early work of camoufleurs such as Andre Mar\'e. 
The use of these colour fields in this way highlights the impact of the viewer's own background on morality and on perception~\cite{kuhn1970structure,popper1972objective}.

Redaction has been used in text-based work such as Jenny Holzer, \emph{Redacted (Top Secret)}, while Ull Hohn, \emph{Sex Painting}, 1987 and Jon Campbell, \emph{Fuck Yeah}, 2016 use obfuscated obscenity as compositional elements. 

Historically, taboo words have progressed from religion, and disease to sex and to slurs against identity groups~\cite{douglas1966purity,pinker2008seven,mcwhorter2024nine}. The assumed intellectual deficiency of swearing has been repeatedly debunked~\cite{reiman2022swearfluency,jay2015taboo}, along with notions of morality~\cite{pinker2017moral,devries2023swearing} and harm~\cite{jay2009offensive}.

Repetition of obscenity, both within a single canvas, and across the entire series is used to reflect on societal desensitisation and the cultural evolution of taboo words, which reflect contemporary morals.  Through repetition I aim to achieve a \emph{semantic drain} where the word loses all meaning~\cite{barthes1957mythologies,derrida1972dissemination}. Going further, I discard the letter structures and use the fragments as autonomous shapes in works that present the viewer with an emergent language where repetition has broken down all existing meaning, establishing completely new meanings~\cite{derrida1972dissemination,deleuze1968difference,baudrillard1981simulacra}.

Obscenity occurs both an autonomous reaction (viewer-as-machine), and a conscious use of language~\cite{jay1999why}. Swearing reveals much about the mechanism of the brain~\cite{bergen2018what}. Through a range of studies, Jay identifies neurological, physiological and cultural factors that modulate the prevalence of swearing. Viewer-as-machine is further activated by exploiting text-completion mechanisms in the brain~\cite{Taylor1953,Reicher1969,Healy1976,GraingerWhitney2004,CohenDehaene2000,Levy2008}. These perceptual mechanisms are culturally dependent~\cite{Bartlett1932,Hall1976,ChuaBolandNisbett2005,MasudaNisbett2001} affecting how we recognise and remember information.

echoing similar processes that underpin all of modern digital computing and communications~\cite{shannon1948}.

\section{Conclusion}
I have responded to the unsettling de-centreing of human intelligence and attendant  othering of machine intelligence by presenting paintings that emphasise the artist-as-machine (through arbitrary rules, optimisations, and mechanistic processes), and viewer-as-machine (via perceptual instabilities, optical illusions, and semantic dissipation that destabilise what the brain and eyes think they ``know''). My paintings embody the Copernican crisis: denying the viewer their accustomed position of perceptual dominance and reminding us that \emph{we are all machines}.

Perceptual instabilities along with repeated, redacted and obfuscated obscenity further reflect on the challenges of human discourse a post-truth society. My paintings participate in the broader cultural negotiation of what it means to be human. Truth and perception have always been unstable, and it is not art's role to offer certainty. Rather, it reveals this fragility, and the possibilities that emerge when  modes of knowing inevitably collapse.

\newpage
\bibliographystyle{jox}
\bibliography{main}
\end{document}
