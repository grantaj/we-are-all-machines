\documentclass[12pt]{article}
\usepackage{graphicx} % Required for inserting images
\usepackage{setspace}
\usepackage[utf8]{inputenc}

\usepackage{titlesec}

\titleformat{\section}  % redefine \section
  [block]               % shape
  {\filcenter\normalfont\scshape} % format: centered, normal weight, small caps
  {}                    % label (empty → unnumbered)
  {0pt}                 % separation between label and title
  {}                    % before-code

\makeatletter
\renewcommand{\maketitle}{%
  \begin{center}
    {\LARGE\scshape\@title \par} % big small caps title
    \vskip 1em
    {\large\scshape\@author \par} % smaller small caps author
    \vskip 1em
    {\normalsize\@date \par} % date in normal size
  \end{center}
}
\makeatother

\usepackage[%
  titleformat=italic,% Titles in italic 
  titleformat=commasep,% A comma between athors and title 
  titleformat=all,% Always show a title (or a short title)
  commabeforerest,% A comma after title 
  ibidem=strict,% 
  citefull=first,% The first citing in full form 
  oxford,% The oxford style
  super,% Footnotes 
  see
]{jurabib}

\interfootnotelinepenalty=10000

\title{We Are All Machines}
\author{Alex Grant}
\date{\today}

\begin{document}
\onehalfspacing
\maketitle

\begin{abstract}
  This paper situates my painting practice within the contemporary
  Copernican crisis, in which human intelligence is de-centred by
  machine intelligence and truth is destabilised by post-truth
  media. I examine how rules, optimisation, and mechanistic processes
  position the artist-as-machine, while perceptual instabilities and
  semantic dissipation confront the viewer-as-machine. Through
  hard-edged abstract oil paintings that exploit colour shift
  phenomena and deploy repeated, redacted obscenity, I expose the
  mechanistic basis of perception and language. These works resist
  confinement to the digital realm, insisting on physical objecthood
  and embodied labour to amplify human provenance. My practice
  participates in the cultural negotiation of what it means to be
  human, reminding us that perception and truth are unstable, and that
  we are all machines.
\end{abstract}

\section{Post Truth Machine Intelligence}
The world is hurtling towards a Copernican crisis, in which human
intelligence is being rapidly de-centred. Regardless of our view of
non-human intelligences, algorithmic entities are having incredible
impact on society. Machine intelligences have
passed~\cite{jones2025largelanguagemodelspass} the Turing
test~\cite{turing1950}, prompting goalpost shifting
responses~\cite{feather2025brainmodelevaluationsneedneuroai}. In these
tests, GPT4.5 tested as human significantly more frequently than
actual humans. In this volatile and uncertain environment, artists'
reactions range from eager adoption of powerful tools as part of an
expanded practice, through to anger, fear and widespread claims AI art
is not ``real art''\footnote{Let's face it -- we can't even agree on
  this for art produced by humans.}. As we approach this crisis, we
face unavoidable dual questions: are machines capable of conscious
behaviour, and conversely, are humans biological machines with
emergent behaviour, long accepted as ``conscious''? In the context of
visual arts we can ask ``is consciousness a prerequisite for
art-making?''

Simultaneously in modern society there is a problem of discerning
truth~\cite{keyes2004posttruth,mcintyre2018posttruth}.  Vested
political and commercial interests use surveillance, big data and vast
computing resources to manipulate attention, perception, in order to
establish their preferred truth.

This is the intellectual and generative context to which I have
responded, manufacturing a series of oil paintings on
canvas. Repetitive artworks utilising optical illusions emphasise the
relative and unstable nature (or nonexistence) of truth, the
unreliability of perception, semantic degradation and
re-emergence. These illusions quantitatively reveal a mechanistic
basis for human visual perception, explained by modern colour science
and physiological understanding of how our eyes and visual processing
circuits work. Hacking the viewer's brain considering optical
illusions as a system vulnerability, I confront the viewer with their
own machine-like nature.

Perception is dependent on our cultural and social background, like
the colour shift phenomenon.  Redacted and obfuscated obscenity is
deployed as a stand-in for control of speech, where each side demands
free speech while simultaneously denying the same to the other side.

The remainder of this paper discusses and contextualises my paintings
within a framework of \emph{formal} qualities, \emph{process} (how I
made the paintings) and their (explicit)
\emph{content}. Representative examples of the paintings are provided
in Figures~\ref{fig:black}--\ref{fig:orange}. Captions are descriptive
and non-titular.

\section{\textsc{Formal Qualities}}
Zylinkska considers how generative art shifts emphasis from material
presence to algorithmic process~\cite{zylinska2020ai}, while Steyerl
extends objecthood to the digital domain~\cite{steyerl2017duty}. In
response, my works are necessarily
\emph{physical}~\cite{fontana1947spatial,merleauponty1964eye}. Extension
of colour fields around the sides of the canvas confirms the spatial
extent of the canvas as object.

These paintings are hard-edged abstractions, which, according to
Greenberg exemplfies self-critical modernism, which denies the image
and focuses on material concerns such as flatness, shape and the
properies of paint.\cite[p. 85-93]{Greenberg1961}. Judd goes further,
collapsing the distinction between sculpture and painting
\emph{object}\cite{Judd1965}, while Alloway, introducing
\emph{Systematic Painting} focuses on repetition, unity, and clarity,
where ``the recurrent image is subject to continuous transformation,
destruction and reconstruction''.\cite[pp. 18-19]{Alloway1975}. In
contrast, Fried critisised such theatricality, demanding work to be
more instantaneous.\cite[p. 12-23]{Fried1968} Relevant to my work are
artists such as Ellsworth Kelly, Ad Reinhardt, Frank Stella and Victor
Vasarely, or more contemporaneously, Odili Donald Odita, Sarah Morris,
Felipe Pantone, and Gabriele Evertz. Evertz carries forward Op Art
practices~\cite{Follin2004EmbodiedVisions,Seitz1965ResponsiveEye}
established by artists such as Bridget
Riley.\cite{Riley2019EyesMind,Riley2019DialoguesOnArt} Focussing on
relative and mechanistic aspects of perception, my paintings rely on
the colour shift phenomenon pioneered by Josef Albers~\cite{albers}
and are colour field paintings in the spirit of his \emph{Homage to
  the Square} series. I predominately use secondary colours to
activate multiple receptors in the viewer's
eye~\cite{HurvichJameson1957,Land1977,SchnapfKraftBaylor1987}.

Executed in 2025, my paintings exist within these art historical
contexts, however hard edges (emphasising artist-as-machine), 
optical effects and perceptual instability are adopted, not for their
own sake\footnote{Although these paintings can also be viewed as
  paintings about painting.}, but in dialogue with AI generated art
and the unreliability of modern public discourse where \emph{everything is
fake}. My use of optical effects relies on understanding of the human
visual processing system, confronting the viewer with their own
machine like qualities (viewer-as-machine).

Gombrich describes the iterative artistic process, with artists
extending stereotyped modes of representation developed by their
predecessors.~\cite[Chapters II, V]{gombrich1960art} Similarly, Popper
argues that perception itself is an active process of interpretation,
involving trial-and-error and learned conventions, where art doesn't
``make sense'' until a viewer has been trained into the relevant
conceptual paradigm.~\cite[Chapter 2]{popper1972objective}.

Echoing Merleau-Ponty~\cite{merleauPonty1962phenomenology} and
Popper~\cite{popper1972objective}, Gabriele Evertz expresses that
``... without the viewer the painting doesn't exist. The viewer brings
the painting to life.''~\cite{evertz09documentary}. While Barthes
would indicate that this is always true~\cite{barthes1977death}, it
especially holds when exploiting optical illusions. Dynamic and
unstable visual effects produce an opportunity for Lee Ufan style
phenomelogical ``encounters''~\cite[p. 52-6]{encounter} produced in
the machinery of the viewer's mind.

According to Kuhn, \emph{Perceptual instability} signals shifting or
contested frameworks.~\cite[p. 64]{kuhn1970structure}
Keyes\cite{keyes2004posttruth} and
McIntyre~\cite{mcintyre2018posttruth} describe our era as
\emph{post-truth}.  Manipulation of public perception occurs through
commodified surveillance capitalism, as described
by Zubhoff\cite{zuboff2019surveillance}. According to
Steyerl\Attention{steyerl2016sea}, attention is a battlfield in which
weaponized data analytics are used by vested media interests. As
elaborated in the next section, my paintings are a carefully
engineered attack on the human visual operating system.

Benjamin~\cite{benjamin1935kunstwerk} argued that \emph{repetition}
provides cultural legitimacy. Arendt argues that similar mechanisms
underpin the power of repeated lies in
politics~\cite{arendt1972lying}. This psychological \emph{illusory
  truth effect} has been studied by Hasher et
al~\cite{hasher1977frequency} and is amplified in the
post-truth~\cite{keyes2004posttruth,mcintyre2018posttruth} circulation
of digital
media~\cite{zuboff2019surveillance}. Foster\cite[p. 29-30]{foster1996return}
discusses how avant garde practices became legible through repetition
by their successors, conferring legitimacy after the fact. Neely
discusses how \emph{repetition legitimises} -- once idea is repeated,
assumed intentionality erases perception of
error.\cite{neely-repetition} I use repetition of simple ideas over
successive canvases as a tool to legitimise an otherwise arbitrary set
of artistic choices.

Iteration, repetition, rotation, and fragmentation are formal devices
employed in my compositions. These transformations are computational
elements. Wolfram~\cite{wolfram1984} proposed that the laws of physics
are actually those of computation, and that mathematical physics
consist of numerical shortcuts for these computations. Complex
physical phenomenae emerge from iteration of simple rules
(one dimensional cellular automata)~\cite{wolfram}. Sara
Walker~\cite{walker2024life} builds on these ideas to propose
\emph{Assembly Theory} which defines life as the propagation of
information through complex, evolving structures. My adoption of
mechanistic ``computational elements'' reflects this idea of
emergent intelligence.

\section{How I made the Paintings}\label{sec:process}
I adopted an engineering approach to the design of these paintings. By
which I mean a process of iterated experiments and computational
optimisations.

Previously working in monochrome, I sought a way to introduce
colour that was non-descriptive and non-aesthetic (or even
anti-aesthetic). This led me to the idea of \emph{functional} use of
colour, with a view to deploying Albers' colour shift phenomenon
as an engineered attack on the human operating system.

Human perception of colour differences~\cite{MacAdam1942} is not
linear\footnote{Equal distances in these spaces do not correspond to
  equal perceptual differences.} in colour spaces such as RGB, CMYK or
HSL~\cite{Luo2001CIECAM02}.  A \emph{perceptually uniform space},
pioneered by Munsell~\cite{Munsell1915} and adopted by the Commission
Internationale de l'\'{E}clairage
(CIE)~\cite{CIE1976,CIE1978Uniform,Luo2001CIEDE2000} is one in which
geometric distance between two colour vectors approximates perceived
difference, allowing Euclidean distance to quantify perceptual shifts.

The CIECAM02 colour appearance
model~\cite{Luo2001CIECAM02,CIE1592004} accounts for
contextual influences such as luminance level, surround adaptation,
and background. Colour is represented in terms of perceptual
attributes: \emph{lightness} $J$ (perceived brightness relative to a
reference white.), \emph{colourfulness} $M$ (perceived intensity or
saturation compared to a reference grey), and \emph{hue angle} $h$
(describing the type of colour sensation e.g., red, blue, green).

I developed a brute-force computational method for identifying
surround colours that cause the greatest perceptual shift of a
stimulus colour. We transform CIECAM02 into CAM02-UCS (Uniform Colour
Space) and quantify differences in perceptual appearance via the
$\Delta E_{CAM02-UCS}$ metric. Exhaustively comparing perceived colour
of the fixed stimulus across candidate surrounds, we identify
surrounds that maximize perceptual shift. This offers an automated,
quantitative analogue to the manual explorations of colour interaction
pioneered by Albers~\cite{albers}. The resulting python computer code
has been made freely
available~\cite{grant2025colourshift}. Remarkably, some of the colour
triads found by Albers using trial-and-error are quite close to
optimum.

Using this tool, I digitally explored colour space, quickly iterating
candidate designs. From these candidates I selected outcomes for
development in oil.

Ad Reinhardt developed self-imposed rules for the production of
art~\cite[p. 203-7]{artasart}. His (rather Greenbergian) rules: no
texture, no brushwork or calligraphy, no colors, no object, no
subject, no matter, are evident in his ``black'' paintings.

Similarly, I adopted my own self-imposed rules for the production of
these paintings: colour choice must be engineered for functional
rather than aesthetic effect, same colours may not touch, letter
fragments should be grouped by colour in opposition to their
semantics, areas should be flat, edges should be hard, paintings
should be executed in a single session.

Reinhardt worked hard to eliminate gesture (he is known to have
methodically brushed out all brush marks, using only a 1 inch
brush~\cite[p. 206]{artasart}). However, I regard this as the artist
laboriously imposing his will over the natural tendency of the medium,
resulting in a painting that speaks very loudly to a human gesture,
denying the will of the paint itself.

In my work, the obvious way to avoid brushstrokes and to have clean,
straight edges is via digital print. However this erases human
presence and would in my view be uninteresting. I am much more
interested in the quiet human gesture detectable in the attempt, but
unavoidable failure to achieve a straight line or a flat area of
colour. It is precisely these ``imperfections'' that convey human
origin and provide physical evidence of human labour in
manufacture.\footnote{I did however use a one inch brush and a 60 inch
  by 60 inch canvas in homage to Reinhardt.} To amplify this effect I
attempted (but sometimes failed) to execute each painting \emph{alla
  prima}, embracing risk of
failure.\cite{rosenberg1952american,moholy1947vision,bois1990painting}.

When emphasising human provenance, one might question the use of
digital tools and computational optimisation. However all art
production involves arbitrary choice of allowable
technologies~\cite{huizinga1938homo,kubler1962shape,flusser2000towards,manovich2001language,steyerl2009poorimage,perec1969disparition}.
When approaching without prejudice the mechanistic basis for machine
and human intelligence, it was natural for me to adopt a hybrid
process.  Canvas, brushes, staples, stretcher bars, oil, pigment are
all technologies. To this list I add digital computation.

According to Benjamin, mechanical reproduction transforms our
perception of authenticity~\cite{benjamin1969art}. In an age of
mechanical \emph{generation} of art, perceived provenance is
significant. Horton et al.~\cite{horton2023bias} conducted experiments
in which AI and human generated art were presented to study
participants, randomly labelled as AI or human. Participants were
asked to score the works by creativity, value, worth and skill. Their
results show that the labels mattered more than the image, with
human-attributed images scoring higher (independent of actual source).
Humans care about art made by humans, not because of what it looks
like, but simply the connection formed by the fact it was \emph{made
  by a human}. Humans value embodied human
labour~\cite{adorno1970aesthetic,singerman1999artist,sennett2008craftsman}. In
another study, Demmer et al.~\cite{demmer2023does} show that machine
generated art can produce emotional responses in humans.

With colour selection and design performed digitally\footnote{Thus my
  conceptual and decision making process inhabits the digital realm,
  while execution and presentation transitions to the physical world.}
within a narrow range of self-imposed rules, execution of the
paintings becomes a mechanistic manufacturing process or
performance~\cite[Sentence
\#28]{LeWitt1969Sentences}\cite{Jones1998BodyArt,LippardChandler1968Dematerialization,lippard1973sixyears}. Taking
an anti-anti-form~\cite{Morris1968AntiForm} stance, the process and
form are equaliy valid.  Presenting the first instances of a
potentially infinite sequence, my paintings imply a space available
for anyone to explore. Additional paintings could be executed by
anyone, similar to Sol LeWitt \emph{Wall Drawings}. In this respect I
adopt Reinhardt's position that ``this painting is my painting if I
paint it... this painting is your painting if you paint
it''~\cite{abstract-painting-1960}. Tehching Hsieh, \emph{Time Clock
  Piece}, On Kawara, \emph{Date Paintings} and Roman Opalka, \emph{One
  to Infinity} are also relevant.

By working in a mechanistic way, producing paintings that highlight
the machine-like qualities of the viewer, I attempt to deal with
machine and human intelligence without prejudice -- pushing back
against an unspoken ``othering'' of AI\footnote{In which anthroponormativity denies cyberneurodivergence and digital queerness}. Human society needs to prepare
for the moral and ethical
questions~\cite{chalmers1996conscious,metzinger2009egotunnel,bostrom2014ethics}
raised by the imminent possibility of machine consciousness. As argued by
Shelvin~\cite{shevlin2023consciousness}, this can probably only be
resolved by shifts in public attitudes and close relationships between
humans and AI.

\section{Explicit Content}
I deploy the word \emph{fuck} as an index for censorship and control
of speech\cite{atkins2006censoring}. This is not aligned with any
ideology, rather an observation that all sides of politics
simultaneously demanding ``free speech'' while aggressively denying it
to their adversaries\cite{lukianoff2023cancelling}.

Empirically optimised redaction\footnote{Via informal survey asking
  participants to view various candidate redaction widths.} has been
used to transform the word into fragments at the boundary of
legibility. The word is simultaneously present and absent,
simultaneously text and shapes. I iteratively progressed from
redaction alone (Figure~\ref{fig:black}), to semantically orthogonal
colour assignments to the fragments (Figure~\ref{fig:colour}), and
finally to semantically disruptive colour fields, similar to but
distinct from the Stroop effect~\cite{stroop1935studies}
(Figure~\ref{fig:green}). This use of colour can be regarded as
camouflage, operating on similar principles as early work of
camoufleurs such as Andre Mar\'e.  The use of colour fields in this
way highlights the impact of the viewer's own background on morality
and on perception~\cite{kuhn1970structure,popper1972objective}.

Redaction has been used in text-based works such as Jenny Holzer,
\emph{Redacted (Top Secret)}, while Ull Hohn, \emph{Sex Painting},
1987 and Jon Campbell, \emph{Fuck Yeah}, 2016 used obfuscated obscenity
as compositional elements.

Historically, taboo words have progressed from religion, and disease
to sex and slurs against identity
groups~\cite{douglas1966purity,pinker2008seven,mcwhorter2024nine}. The
assumed intellectual deficiency of swearing has been repeatedly
debunked~\cite{reiman2022swearfluency,jay2015taboo}, along with
notions of morality~\cite{pinker2017moral,devries2023swearing} and
harm~\cite{jay2009offensive}.

Repetition of obscenity, within a single canvas, and across the series
mirrors desensitisation and cultural evolution of taboo words.
Through repetition I aim to achieve a \emph{semantic drain} where the
word loses all
meaning~\cite{barthes1957mythologies,derrida1972dissemination}. Going
further, I discard the letter structures and use the fragments as
autonomous shapes in works (e.g. Figure~\ref{fig:orange}) that present
the viewer with an emergent language where repetition has broken down
all existing meaning, establishing completely new
meanings~\cite{derrida1972dissemination,deleuze1968difference,baudrillard1981simulacra}.

Swearing reveals much about the mechanism of the
brain~\cite{bergen2018what}. Obscenity occurs both an autonomous
reaction (viewer-as-machine), and a conscious use of
language~\cite{jay1999why}.  Through a range of studies, Jay
identifies neurological, physiological and cultural factors that
modulate swearing prevalence. Viewer-as-machine is further activated
by exploiting text-completion mechanisms in the
brain~\cite{Taylor1953,Reicher1969,Healy1976,GraingerWhitney2004,CohenDehaene2000,Levy2008}. These
perceptual mechanisms are culturally
dependent~\cite{Bartlett1932,Hall1976,ChuaBolandNisbett2005,MasudaNisbett2001}
affecting how we recognise and remember information.

Similar processes underpin all of modern digital computing and
communications~\cite{shannon1948}.

\section{Conclusion}
I have responded to the unsettling de-centreing of human intelligence
and attendant othering of machine intelligence through hard-edged oil
paintings paintings that emphasise the artist-as-machine (through
arbitrary rules, optimisations, and mechanistic processes), and
viewer-as-machine (via perceptual instabilities, optical illusions,
and semantic dissipation that destabilise what the brain and eyes
think they ``know''). 

Perceptual instabilities along with repeated, redacted and obfuscated
obscenity further reflect on the challenges of human discourse in a
post-truth society. My paintings participate in the broader cultural
negotiation of what it means to be human. Truth and perception have
always been unstable, and it is not art's role to offer
certainty. Rather, it reveals this fragility, and the possibilities
that emerge when modes of knowing inevitably collapse.

My paintings embody the AI Copernican crisis:
denying the viewer their accustomed position of perceptual dominance
and reminding us that \emph{we are all machines}.

\newpage
\begin{figure}[htbp]
  \centering
  \includegraphics*[width=0.9\textwidth]{black.jpg}
  \caption{Redacted text, monochrome -- oil on canvas, 48 x 36 inches.}
  \label{fig:black}
\end{figure}

\begin{figure}[htbp]
  \centering
  \includegraphics*[width=0.9\textwidth]{colour.jpg}
  \caption{Redacted text, coloured fragments -- oil on canvas, 48 x 36 inches.}
  \label{fig:colour}
\end{figure}

\begin{figure}[htbp]
  \centering
  \includegraphics*[width=0.9\textwidth]{green.jpg}
  \caption{Colour shift -- oil on canvas, 48 x 36 inches.}
  \label{fig:green}
\end{figure}

\begin{figure}[htbp]
  \centering
  \includegraphics*[width=0.9\textwidth]{orange.jpg}
  \caption{Emergent semantics -- oil on canvas, 30 x 30 inches.}
  \label{fig:orange}
\end{figure}

\newpage
\bibliographystyle{jox}
\bibliography{main}
\end{document}
