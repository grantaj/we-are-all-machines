\documentclass[12pt]{article}
\usepackage{graphicx} % Required for inserting images
\usepackage{setspace}
\usepackage[utf8]{inputenc}
\nocite{*}
\usepackage[%
  titleformat=italic,% Titles in italic 
  titleformat=commasep,% A comma between athors and title 
  titleformat=all,% Always show a title (or a short title)
  commabeforerest,% A comma after title 
  ibidem=strict,% 
  citefull=first,% The first citing in full form 
  oxford,% The oxford style
  super,% Footnotes 
  see
]{jurabib}

\interfootnotelinepenalty=10000

\title{We Are All Machines}
\author{Alex Grant}
\date{\today}

\begin{document}
\onehalfspacing
\maketitle

We are in the midst of a Copernican crisis in which human intelligence is
being rapidly de-centred by machine intelligence. Large language
models already pass Turing tests more frequently than
humans\cite{jones2025largelanguagemodelspass,turing1950,feather2025brainmodelevaluationsneedneuroai},
provoking cultural anxiety and questions about whether machine
generated art is ``real art.'' My response is to ask whether
consciousness is a requirement to make art, and conversely whether
humans, as biological machines, merely exhibit emergent behaviour
under the doxa of consciousness.

Simultaneously,
post-truth\cite{keyes2004posttruth,mcintyre2018posttruth} media and
other vested interests weaponise online surveillance and data
analytics to manipulate attention and
perception\cite{zuboff2019surveillance,steyerl2016sea}. My work
inhabits the intersection of these two crises.

I have engineered and manufactured a series of hard-edged abstract oil
paintings in which optical illusions expose the mechanistic basis of
perception. I use the colour shift phenomenon as an attack on the
human visual processing system, confronting the viewer with their own
machine-like nature.

Formally, the works resist confinement to the digital
realm\cite{zylinska2020ai,steyerl2017duty}, insisting on physical
objecthood\cite{fontana1947spatial,merleauponty1964eye}. They draw on
the lineage of hard-edge
abstraction\cite{Greenberg1961,Alloway1975,Judd1965,Fried1968,Follin2004EmbodiedVisions}
and Op Art\cite{Riley2019EyesMind,Seitz1965ResponsiveEye}, extending
Albers’s exploration of colour shift\cite{albers} through contemporary
colour
science\cite{HurvichJameson1957,Land1977,SchnapfKraftBaylor1987}. Perception
is framed as an interpretive process structured by
conventions\cite{popper1972objective,gombrich1960art}, and completed
mecahnistically in the viewer's
mind\cite{merleauPonty1962phenomenology,barthes1977death,evertz09documentary}. Perceptual
instability signals contested ontologies\cite{kuhn1970structure}, and
repetition both
legitimises\cite{benjamin1935kunstwerk,arendt1972lying,hasher1977frequency}
and destabilises meaning\cite{zuboff2019surveillance}.

My process adopts an engineering methodology of iteration and
computational optimisation. Using CIECAM02 colour models and the
uniform perceptual space
CAM02-UCS\cite{Luo2001CIECAM02,CIE1592004,CIE1976,CIE1978Uniform,Luo2001CIEDE2000},
I developed software\cite{grant2025colourshift} to brute-force colour
surrounds that maximise perceptual shift of a stimulus colour. This
automates Albers’s trial-and-error approach\cite{albers}. I generate
candidate designs digitally, down-select and then mechanistically
realise the outcomes in oil on canvas under self-imposed rules
recalling Reinhardt\cite{artasart}: functional not aesthetic colour,
flat areas, hard edges, and single-session execution. In contrast to the more expedient route of digital print, I embrace the failure of perfect execution, preserving
human presence\cite{rosenberg1952american,bois1990painting}. In this way my
work responds to human preference for human authorship and
provenance\cite{benjamin1969art,horton2023bias,demmer2023does},
and emphasises the cultural weight of embodied
labour\cite{adorno1970aesthetic,sennett2008craftsman}.

My compositions use redacted and fragmented versions of the word
\emph{fuck} as an index of contested
speech\cite{atkins2006censoring,lukianoff2023cancelling} in our
post-truth world. Through empirically tuned redaction, orthogonal
colour assignments and disruptive colour fields, the text hovers
between legibility and abstraction, drawing on camouflage and
Stroop-like effects. This engages with contemporary research on
swearing’s cognitive, cultural and neurological
bases\cite{reiman2022swearfluency,jay2015taboo,pinker2017moral,bergen2018what}. Repetition
produces semantic
drain\cite{barthes1957mythologies,derrida1972dissemination}, to the
point of new emergent
meaning\cite{deleuze1968difference,baudrillard1981simulacra}.

Summarising: I made paintings that foreground the artist-as-machine
(through rules, optimisations and mechanistic processes) and the
viewer-as-machine (through optical illusion and semantic dissipation).

Perceptual instabilities along with repeated, redacted and obfuscated
obscenity reflect on the challenges of human discourse in post-truth
society. My paintings participate in the broader cultural negotiation
of what it means to be human. Truth and perception have always been
unstable, and it is not art's role to offer certainty. Rather, it
reveals this fragility, and the possibilities that emerge when modes
of knowing inevitably collapse.

My paintings embody the current Copernican crisis of artificial
intelligence: denying the viewer their accustomed position of
perceptual dominance and reminding us that \emph{we are all machines}.

\newpage
\bibliographystyle{jox}
\bibliography{main}
\end{document}
